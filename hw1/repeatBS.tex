\documentclass{article}

\usepackage{listings}
\usepackage{indentfirst}
\usepackage{verbatim}
\usepackage{amsmath, amsthm, amssymb}
\usepackage{stmaryrd}
\usepackage{graphicx}

\usepackage[utf8]{inputenc}
\usepackage[english,russian]{babel}
\usepackage[T2A]{fontenc}

\lstdefinelanguage{llang}{
keywords={skip, do, while, read, write, if, then, else},
sensitive=true,
%%basicstyle=\small,
commentstyle=\scriptsize\rmfamily,
keywordstyle=\ttfamily\underbar,
identifierstyle=\ttfamily,
basewidth={0.5em,0.5em},
columns=fixed,
fontadjust=true,
literate={->}{{$\to$}}1
}

\lstset{
language=llang
}

\newcommand{\aset}[1]{\left\{{#1}\right\}}
\newcommand{\term}[1]{\mbox{\texttt{\bf{#1}}}}
\newcommand{\cd}[1]{\mbox{\texttt{#1}}}
\newcommand{\sembr}[1]{\llbracket{#1}\rrbracket}
\newcommand{\conf}[1]{\left<{#1}\right>}
\newcommand{\fancy}[1]{{\cal{#1}}}

\newcommand{\trule}[2]{\frac{#1}{#2}}
\newcommand{\crule}[3]{\frac{#1}{#2},\;{#3}}
\newcommand{\withenv}[2]{{#1}\vdash{#2}}
\newcommand{\trans}[3]{{#1}\xrightarrow{#2}{#3}}
\newcommand{\ctrans}[4]{{#1}\xrightarrow{#2}{#3},\;{#4}}
\newcommand{\llang}[1]{\mbox{\lstinline[mathescape]|#1|}}
\newcommand{\pair}[2]{\inbr{{#1}\mid{#2}}}
\newcommand{\inbr}[1]{\left<{#1}\right>}
\newcommand{\highlight}[1]{\color{red}{#1}}
\newcommand{\ruleno}[1]{\eqno[\textsc{#1}]}
\newcommand{\inmath}[1]{\mbox{$#1$}}

\newcommand{\NN}{\mathbb N}
\newcommand{\ZZ}{\mathbb Z}

\begin{document}

\textbf{Домашняя работа по курсу "Статический анализ программ"}

\hfill От 17.09.13

\hfill Выполнил cтудент 545 гр. Подкопаев Антон

\hrule
\vspace{1.2cm}

\section{Описание семантики Repeat-Until без использования других операторов \label{withoutOper}}

$$
\crule{\trans{c}{\mbox{$S$}}{c^\prime}}
      {\trans{c}{\llang{repeat $\;S\;$ until $\;e\;$}}{c^\prime}}
      {\sembr{e}\;{c^{\prime}}.s\ne0}
\ruleno{Repeat1-true$_{bs}$}
$$

$$
\crule
		{\trans{c}{\mbox{$S$}}{c^{\prime\prime}},\;\trans{c^{\prime\prime}}{\llang{repeat $\;S\;$ until $\;e\;$}}{c^\prime}}
		{\trans{c}{\llang{repeat $\;S\;$ until $\;e\;$}}{c^\prime}}
		{\sembr{e}\;{c^{\prime\prime}}.s=0}
\ruleno{Repeat1-false$_{bs}$}
$$

\section{Описание семантики Repeat-Until через другие операторы \label{withOper}}

$$
\crule
		{
			\trans{c}
				{\mbox{$S$}}
			{c^{\prime\prime}}
			,\;
			\trans{c^{\prime\prime}}
				{\llang{while $\;e=0\;$ do $\;S\;$}}
			{c^{\prime}}			
		}
		{
			\trans{c}
				{\llang{repeat $\;S\;$ until $\;e\;$}}
			{c^\prime}
		}
		{}
\ruleno{Repeat2$_{bs}$}
$$

\section{Доказательство эквивалентности представленных семантик}
% Предположим, что из конфигурации $c$ под действием семантики BS, дополненной семантическими правилами $Repeat1-true_{bs}$ и $Repeat1-false_{bs}$, оператора \llang{repeat $\;S\;$ until $\;e\;$}, по\begin{figure}лучается конфигурация $c^\prime$:

Пусть с использованием правил $Repeat1-true_{bs}$ и $Repeat1-false_{bs}$ было произведено преобразование рис. \ref{ccprime}.

\begin{figure}[h!]
$$
\trans{c}{\llang{repeat $\;S\;$ until $\;e\;$}}{c^\prime}
$$
\caption{Преобразование с помощью $Repeat1_{bs}$}
\label{ccprime}
\end{figure}

Тогда если была получена финальная конфигурация $c^\prime$, то дерево вывода имеет некоторую конечную высоту --- $N$. При этом считаем, что в дереве вывода не раскрывается оператор $S$. Так как применение $Repeat1-false_{bs}$ не может быть листом в дереве вывода, то последнее примененное правило --- $Repeat1-true_{bs}$.
Также понятно, что применение $Repeat1-true_{bs}$ не может быть не последним. Таким образом дерево вывода может быть записано как на рис. \ref{ccprimeTree1}.

\begin{figure}[h!]

$$
\crule
	{
		\trans{c}{\mbox{$S$}}{c_1}
		,\;
		% \trans{c_1}{\llang{repeat $\;S\;$ until $\;e\;$}}{c^\prime}			
		\crule
			{
				\trans{c_1}{\mbox{$S$}}{c_2}
				,\;
				% \trans{c_2}{\llang{repeat $\;S\;$ until $\;e\;$}}{c^\prime}	
				\crule
				{
					\crule
					{\trans{c_N}{\mbox{$S$}}{c^\prime}}
					{...}
					{\sembr{e}\;{c_N}.s\ne0}
					% \ruleno{Repeat1-true$_{bs}$}
				}
				{
					\trans{c_2}{\llang{repeat $\;S\;$ until $\;e\;$}}{c^\prime}	
				}
				{}
			}
			{\trans{c_1}{\llang{repeat $\;S\;$ until $\;e\;$}}{c^\prime}}
			{\sembr{e}\;c_2.s=0}
		% $\ruleno{Repeat1-false$_{bs}$}$
	}
	{
		\trans{c}{\llang{repeat $\;S\;$ until $\;e\;$}}{c^\prime}		
	}
	{\sembr{e}\;c_1.s=0}
\ruleno{Repeat1-false$_{bs}$}
$$
\label{ccprimeTree1}
\caption{Дерево вывода, построенное с помощью $Repeat1_{bs}$}
\end{figure}

При построении дерева вывода с помощью $Repeat2_{bs}$ на первых итерациях получается аналогичное дерево (см. рис. \ref{ccprimeTree2}).

\begin{figure}[h!]

$$
\crule
	{
		\trans{c}{\mbox{$S$}}{c^\prime_1}
		,\;
		\crule
			{
				\trans{c^\prime_1}{\mbox{$S$}}{c^\prime_2}
				,\;
				\crule
				{...}
				{
					\trans{c^\prime_2}{\llang{while $\;e=0\;$ do $\;S\;$}}{c^\prime}	
				}
				{}
			}
			{\trans{c^\prime_1}{\llang{while $\;e=0\;$ do $\;S\;$}}{c^\prime}}
			{\sembr{e}\;c^\prime_2.s=0}
	}
	{
		\trans{c}{\llang{while $\;e=0\;$ do $\;S\;$}}{c^\prime}		
	}
	{\sembr{e}\;c^\prime_1.s=0}
\ruleno{Repeat2$_{bs}$}
$$
\label{ccprimeTree2}
\caption{Дерево вывода, построенное с помощью $Repeat2_{bs}$}
\end{figure}

Очевидно, что конфигурации $c_i$ и $c'_i$ совпадают для всех соответствующих $i$.
Тогда из того, что $\sembr{e}\;{c_N}.s\ne0$ следует $\sembr{e}\;{c^\prime_N}.s\ne0$.
Соотвественно, дерево вывода, построенное с помощью $Repeat2_{bs}$ будет выглядеть как изображенное на рис. \ref{ccprimeTree2Full}.

\begin{figure}[h!]

$$
\crule
	{
		\trans{c}{\mbox{$S$}}{c_1}
		,\;
		\crule
			{
				\trans{c_1}{\mbox{$S$}}{c_2}
				,\;
				\crule
				{
					\crule
					{\trans{c_N}{\mbox{$S$}}{c^\prime}}
					{...}
					{\sembr{e}\;{c_N}.s\ne0}
				}
				{
					\trans{c_2}{\llang{while $\;e=0\;$ do $\;S\;$}}{c^\prime}	
				}
				{}
			}
			{\trans{c_1}{\llang{while $\;e=0\;$ do $\;S\;$}}{c^\prime}}
			{\sembr{e}\;c_2.s=0}
	}
	{
		\trans{c}{\llang{while $\;e=0\;$ do $\;S\;$}}{c^\prime}		
	}
	{\sembr{e}\;c_1.s=0}
\ruleno{Repeat2$_{bs}$}
$$
\label{ccprimeTree2Full}
\caption{Полное дерево вывода, построенное с помощью $Repeat2_{bs}$}
\end{figure}


\end{document}